%
% 6.006 problem set 0 solutions template
%
\documentclass[12pt,twoside]{article}

\input{macros-sp20}
\newcommand{\theproblemsetnum}{0}

\title{6.006 Problem Set 0}

\begin{document}

\handout{Problem Set \theproblemsetnum}
%\setlength{\parindent}{0pt}
\medskip\hrulefill\medskip

{\bf Name:} Your Name

\medskip\hrulefill

%%%%%%%%%%%%%%%%%%%%%%%%%%%%%%%%%%%%%%%%%%%%%%%%%%%%%
% See below for common and useful latex constructs. %
%%%%%%%%%%%%%%%%%%%%%%%%%%%%%%%%%%%%%%%%%%%%%%%%%%%%%

% Some useful commands:
% $f(x) = \Theta(x)$
% $T(x, y) \leq \log(x) + 2^y + \binom{2n}{n}$
% \ttt{code\_function}


% You can create unnumbered lists as follows:
% \begin{itemize}
%     \item First item in a list
%         \begin{itemize}
%             \item First item in a list
%                 \begin{itemize}
%                     \item First item in a list
%                     \item Second item in a list
%                 \end{itemize}
%             \item Second item in a list
%         \end{itemize}
%     \item Second item in a list
% \end{itemize}

% You can create numbered lists as follows:
% \begin{enumerate}
%     \item First item in a list
%     \item Second item in a list
%     \item Third item in a list
% \end{enumerate}

% You can write aligned equations as follows:
% \begin{align}
%     \begin{split}
%         (x+y)^3 &= (x+y)^2(x+y) \\
%                 &= (x^2+2xy+y^2)(x+y) \\
%                 &= (x^3+2x^2y+xy^2) + (x^2y+2xy^2+y^3) \\
%                 &= x^3+3x^2y+3xy^2+y^3
%     \end{split}
% \end{align}

% You can create grids/matrices as follows:
% \begin{align}
%     A =
%     \begin{bmatrix}
%         A_{11} & A_{21} \\
%         A_{21} & A_{22}
%     \end{bmatrix}
% \end{align}

\begin{problems}

\problem  $A = \{ 1, 6, 12, 13, 9 \}$, and $B = \{ 3, 6, 12, 15 \}$. Therefore

\begin{problemparts}
\problempart $A \cap B = \{ 6, 12 \}$
\problempart $|A \cup B| = |\{ 1, 3, 6, 9, 12, 13, 15 \}| = 7$
\problempart $|A - B| = |\{ 1, 9, 13 \}| = 3$
\end{problemparts}

\problem  % Problem 2

\begin{problemparts}
\problempart
$$
E[X] = \sum_{i = 0}^3 iP(\text{flip }i\text{ heads}) = 0\left(\frac{1}{8}\right) + 1\left(\frac{3}{8}\right) + 2\left(\frac{3}{8}\right) + 3\left(\frac{1}{8}\right) = \frac{3}{2}
$$
\problempart Let $Z$ be the random variable representing the outcome of rolling one fair six-sided dice. Then $Y = X^2$, and by the properties of expectation,
$$
E[Y] = E[Z^2] = E[(Z - E(Z)^2)] + E[Z]^2 = \Var(Z) + E[Z]^2
$$
Since $Z$ is the discrete uniform distribution on $[1, 6]$, 
$$
E[Z] = \frac{1 + 6}{2} = \frac{7}{2}
$$
$$
\Var(Z) = \frac{(6 - 1 + 1)^2 - 1}{12} = \frac{35}{12}
$$
so 
$$
E[Y] = \frac{35}{12} + \left(\frac{7}{2}\right)^2 = \frac{91}{6}
$$
\problempart Using the linearity of expectation,
$$
E[X + Y] = E[X] + E[Y] = \frac{3}{2} + \frac{91}{6} = \frac{50}{3}
$$
\end{problemparts}

\problem $A = 600/6 = 100$ and $B = 60 \mod 42 = 18 \mod 42$, so
$$
B = 18 + 42k
$$
for some $k \in \mathbb{Z}$.

\begin{problemparts}
\problempart Showing that $A \equiv B \mod 2$ is equivalent to showing that $A$ and $B$ are both even, which is true.
\problempart $A \equiv 1 \mod 3$, and since $18$ and $42$ are both divisible by $3$, $B \equiv 0 \mod 3$ so this is false.
\problempart $A \equiv 0 \mod 4$, and $B \equiv 2 + 2k \mod 4$; the right hand side is $0$ if and only if $k$ is odd:

$\Rightarrow$: if $2 + 2k \equiv 0 \mod 4$, then for some $m \in \mathbb{Z}$, $2 + 2k = 4m$ which implies that $k = 2m - 1$, so $k$ is odd.

$\Leftarrow$: if $k$ is odd, then $k = 2m + 1$ for some $m \in \mathbb{Z}$, so 
$$
2 + 2k = 2 + 2(2m + 1) = 4m + 4 \equiv 0 \mod 4.
$$
\end{problemparts}

\problem  % Problem 4

Base case: for $n = 1$,
$$
\left[\frac{n(n + 1)}{2}\right]^2 = \left[\frac{1(2)}{2}\right]^2 = 1^3 = \sum_{i = 1}^n i^3.
$$

Inductive step: assume that
$$
\sum_{i = 1}^n i^3 = \left[\frac{n(n + 1)}{2}\right]^2
$$
for $n \leq N$. Then
$$
\sum_{i = 1}^{N + 1} i^3 = (N + 1)^3 + \sum_{i = 1}^N i^3 = (N + 1)^3 + \left[\frac{N(N + 1)}{2}\right]^2
$$
Combining terms on the right hand side,
$$
(N + 1)^3 + \frac{N^2(N + 1)^2}{4} = \frac{4(N + 1)^3 + N^2(N + 1)^2}{4} = \frac{(4(N + 1) + N^2)(N + 1)^2}{4}
$$
The term $4(N + 1) + N^2$ is $(N + 2)^2$, so this simplifies to
$$
\frac{(N + 2)^2(N + 1)^2}{4} = \left[\frac{(N + 1)(N + 2)}{2}\right]^2
$$
showing that for $n = N + 1$,
$$
\sum_{i = 1}^n i^3 = \left[\frac{n(n + 1)}{2}\right]^2
$$
\newpage
\problem  Induct on $|V|$:

Base case: suppose $|V|$ = 1, so $|E| = 0$. Then $G$ is the graph with one vertex and no edges, which is acylclic.

Inductive step: assume that every connected undirected graph $G = (V, E)$ with $|E| = |V| - 1$ is acyclic for $|V| \leq N$. Suppose that $G$ is a connected undirected graph with $|V| = N + 1$ and $|E| = N$, and suppose that $G$ contains a $k$-cycle for some $k \leq N + 1$. Note that $k$ cannot equal $N + 1$, because by definition a $k$-cycle contains $k$ vertices and $k$ edges, and $G$ has $N$ edges, so it cannot contain an $N + 1$-cycle. Then there exists some $v \in V$ that is not part of this cycle. More specifically, we can find $v \in V$ that is not part of any cycle, and whose degree is 1: since $|E| = |V| - 1$, then
$$
\sum_{v \in V} \deg v = 2 |E| = 2(|V| - 1)
$$
where $\deg v \geq 1$ for all $v \in V$ since $G$ is connected. If $\deg v \geq 2$ for all $v \in V$, then the left hand side would be at least $2|V|$, which is a contradiction. Then there exists at least one $v' \in V$ whose degree is exactly 1. Additionally, we know that $v'$ cannot be part of any cycle, because if it was its degree would be at least 2.

Let $G'$ be the graph obtained by removing $v'$ and its edge from $G$. Then $G'$ is a connected, undirected graph with $|V| = N$, $|E| = N - 1$, containing a cycle, which is a contradiction. This shows that $G$ cannot contain a cycle.

\vfill
\problem  % Problem 6
Submit your implementation {\small\url{alg.mit.edu}}.

\begin{lstlisting}
def count_long_subarray(A):
    '''
    Input:  A     | Python Tuple of positive integers
    Output: count | number of longest increasing subarrays of A
    '''
    count = 0
    ##################
    # YOUR CODE HERE #
    ##################
    # track the current longest increasing subarray
    # and the length of the current increasing subarray
    longest_subarray_length = 0
    current_subarray_length = 0

    # iterate through the indices: for each i from 1 ... len(A) - 1,
    # if A[i - 1] < A[i], we're in an increasing subarray
    # increment the current subarray length by 1 (or 2 if it's 0)
    # compare to the longest subarray length: if it's >, update and set count to 1
    # if it's ==, increment count
    # if A[i - 1] >= A[i], then we're out of the subarray - set current length to 0
    for i in range(1, len(A)):
        if (A[i - 1] < A[i]):
            current_subarray_length += 2 if current_subarray_length == 0 else 1

            if (current_subarray_length == longest_subarray_length):
                count += 1
            elif (current_subarray_length > longest_subarray_length):
                longest_subarray_length = current_subarray_length
                count = 1
        else:
            current_subarray_length = 0
    return count
\end{lstlisting}

\end{problems}

\end{document}
